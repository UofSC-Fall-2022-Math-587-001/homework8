\documentclass[12pt]{amsart}
\usepackage{amsmath}
\usepackage{amsthm}
\usepackage{amsfonts}
\usepackage{amssymb}
\usepackage[margin=1in]{geometry}
\usepackage{hyperref}
\hypersetup{
    colorlinks=true,
    linkcolor=blue
}

\theoremstyle{definition}
\newtheorem{theorem}{Theorem}[section]
\newtheorem{lemma}[theorem]{Lemma}
\newtheorem{definition}[theorem]{Definition}
\newtheorem{corollary}[theorem]{Corollary}
\newtheorem{proposition}[theorem]{Proposition}
\newtheorem{conjecture}[theorem]{Conjecture}
\newtheorem{remark}[theorem]{Remark}
\newtheorem{example}[theorem]{Example}
\newtheorem{problem}[theorem]{Problem}
\newtheorem{notation}[theorem]{Notation}
\newtheorem{question}[theorem]{Question}
\newtheorem{caution}[theorem]{Caution}

\begin{document}

\title{Homework 7}

\maketitle

For this week, please answer the following questions from the text. 
I've copied the problem itself below and the question numbers for 
your convenience. 

\begin{enumerate}
	\item (3.1) Solve the following congruences. 
	\begin{enumerate}
		\item $x^{19} = 36 \mod 97$ 
		\item $x^{137} = 428 \mod 541$ 
		\item $x^{73} = 614 \mod 1159$ 
		\item $x^{751} = 677 \mod 8023$ 
		\item $x^{38993} = 328047 \mod 401227$ (Hint: $402117 = 
			608 \cdot 661$)
	\end{enumerate}
	\item (3.4) Recall from Sect. 1.3 that \textit{Euler's phi function} 
		$\phi(N)$ is defined by 
	\begin{displaymath}
		\phi(N) = \# \lbrace 0 \leq k < N \mid \operatorname{gcd}(
		k,N) = 1 \rbrace
	\end{displaymath}
		In other words, $\phi(N)$ is the number of integers between 
		$0$ and $N-1$ that are relatively prime to $N$, or 
		equivariantly, the number of elements of $\mathbb{Z}/N\mathbb{Z}$ 
		that have inverses modulo $N$. 
	\begin{enumerate}
		\item Compute the values of $\phi(6), \phi(9), \phi(15)$, 
			and $\phi(17)$. 
		\item If $p$ is prime, what is the value of $\phi(p)$? 
		\item Prove \textit{Euler's formula} for all $a$ satisfying 
			$\operatorname{gcd}(a,N) = 1$
		\begin{displaymath}
			a^{\phi(N)} = 1 \mod N 
		\end{displaymath}
			(Hint: Mimic the proof of Fermat's little theorem 
			(Theorem 1.24), but instead of looking at all the 
			multiples of $a$ as was done in (1.8), just take the 
			multiples $ka$ of $a$ for values of $k$ satisfying 
			$\operatorname{gcd}(k,N)=1$). 
	\end{enumerate}
	\item (3.7) Alice publishes her RSA public key: modulus $N = 2038667$  
		and exponent $e=103$. 
	\begin{enumerate}
		\item Bob wants to send Alice the message $m=892383$. What 
			ciphertext does Bob send to Alice?
		\item Alice knows that her modulus factors into the produce of 
			primes, one of which is $p=1301$. Find a decryption 
			exponent $d$ for Alice. 
		\item Alice receives the ciphertext $317730$ from Bob. Decrypt 
			the message.
	\end{enumerate}
	\item (3.9) For each of the given values of $N = pq$ and $(p-1)(q-1)$, use 
		the method described in Remark 3.11 to determine $p$ and $q$. 
	\begin{enumerate}
		\item $N = pq = 325717$ and $(p-1)(q-1) = 351520$. 
		\item $N = pq = 77083921$ and $(p-1)(q-1) = 77066212$. 
		\item $N = pq = 109404161$ and $(p-1)(q-1) = 109380612$. 
		\item $N = pq = 172205490419$ and $(p-1)(q-1) = 172204660344$. 
	\end{enumerate}
	\item (3.15) Use the Miller-Rabin test on each of the following numbers. 
		In each case, either provide a Miller-Rabin witness for the 
		compositeness of $n$, or conclude that $n$ is probably prime 
		by providing $10$ numbers that not Miller-Rabin witnesses for 
		$n$. 
	\begin{enumerate}
		\item $n=1105$ 
		\item $n=294409$ 
		\item $n=294439$ 
		\item $n=118901509$ 
		\item $n=118901521$ 
		\item $n=118901527$ 
		\item $n=118915387$ 
	\end{enumerate}
\end{enumerate}
\end{document}
